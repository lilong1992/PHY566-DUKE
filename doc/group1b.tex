\documentclass{article}
\usepackage{geometry}
\usepackage{graphicx}
\usepackage{caption}
\usepackage{subcaption}
\usepackage{amsmath}
\geometry{left=1.8cm,right=1.8cm,top=1.5cm,bottom=1.5cm}
\parindent=0pt

\begin{document}
\title{\textbf{Computational Physics Group Assignment 1}}
\author{Christopher Flower, Long Li, Shen Yan, Wenzhe Yu}
\maketitle

\section{Introduction}
In this work, the 2D random walk, diffusion equation, and mixing of two gases are studied by Python programs.

\section{2D Random Walk}
A 2-dimensional random walker is a point continuously taking steps of unit length in $\pm x$ or $\pm y$ direction on a discrete square lattice. Random walkers taking up to 100 steps are simulated, then the properties of 2D random walk are investigated by averaging over $10^4$ walks.\\
\\
The average of x-coordinate $<x_n>$, of the square of x-coordinate $<(X_n)^2>$, and of the square distance $<(r_n)^2>$ of random walks with different steps are plotted in figure~\ref{random} (a), (b), and (c), respectively. $<x_n>$ oscillates around zero, consistent with the theoretical expectation. $<(x_n)^2>$, by contrast, increases with steps with a slope of $\frac{1}{2}$.\\
\\
Figure~\ref{random} (c) shows that $<(r_n)^2>$ increases with steps as well, with a slope of $1$. This can be explained by the symmetry between  $<(y_n)^2>$ and $<(x_n)^2>$, leading the slope of $<(r_n)^2>$ be $\frac{1}{2}+\frac{1}{2} = 1$.

\begin{figure}[h!]
\centering
\begin{subfigure}[b]{0.32\textwidth}
\includegraphics[width=\textwidth]{xaverage.pdf}
\caption{$<x_n>$ versus steps n}
\end{subfigure}
\begin{subfigure}[b]{0.32\textwidth}
\includegraphics[width=\textwidth]{x2average.pdf}
\caption{$<(x_n)^2>$ versus steps n}
\end{subfigure}
\begin{subfigure}[b]{0.32\textwidth}
\includegraphics[width=\textwidth]{r2average.pdf}
\caption{$<(r_n)^2>$ versus steps n}
\end{subfigure}
\caption{$<x_n>$, $<(x_n)^2>$, and $<(r_n)^2>$ of 2D random walk.}
\label{random}
\end{figure}

\section{Diffusion Equation}
\textbf{a)} Consider the 1D Normal Distribution

\begin{equation}
\rho(x,t) = \frac{1}{\sqrt{2\pi \sigma^2 (t)}} e^{-\frac{x^2}{2\sigma^2(t)}}
\end{equation}

The spatial expectation value $<x^2(t)>$ can be computed by

\begin{align*}
<x^2(t)> & = \int_{-\infty}^{\infty} x^2 \rho(x,t) dt \\
& = \frac{1}{\sqrt{2\pi \sigma^2 (t)}} \int_{-\infty}^{\infty} x^2 e^{-\frac{x^2}{2\sigma^2(t)}} dx \\
& = \frac{1}{\sqrt{2\pi \sigma^2 (t)}} \frac{1}{2} (\sqrt{2} \sigma(t))^3 \sqrt{\pi} \\
& = \sigma^2 (t)
\end{align*}

Here $\int_{-\infty}^{\infty} x^{2k} e^{-\frac{x^2}{a^2}} dx = \frac{(2k+1)!!}{(2k+1)2^k} a^{2k+1} \sqrt{\pi}\ (k=0,1,2,...)$ is used.\\
\\
\textbf{b)} The diffusion equation with a constant diffusion coefficient has the following form:

\begin{equation}
\frac{\partial u (r,t)}{\partial t} = D \nabla^2 u(r,t)
\end{equation}

In one-dimensional case, equation (2) becomes

\begin{equation}
\frac{\partial u (r,t)}{\partial t} = D \frac{\partial^2 u(x,t)}{\partial x^2}
\end{equation}

The first derivative in time and second derivative in space can be approximated by the finite difference:

\begin{align*}
\frac{\partial u (r,t)}{\partial t} & = \frac{u(x,t+\Delta t) - u(x,t)}{\Delta t} \\
\frac{\partial^2 u(x,t)}{\partial x^2} & = \frac{u(x+\Delta x,t)+u(x-\Delta x,t)-2u(x,t)}{(\Delta x)^2}
\end{align*}

Then equation (3) can be rewritten as

\begin{equation*}
\frac{u(x,t+\Delta t) - u(x,t)}{\Delta t} = D \cdot \frac{u(x+\Delta x,t)+u(x-\Delta x,t)-2u(x,t)}{(\Delta x)^2}
\end{equation*}

or equivalently,

\begin{equation}
u(x,t+\Delta t) = u(x,t) + D \cdot \Delta t \cdot \frac{u(x+\Delta x,t)+u(x-\Delta x,t)-2u(x,t)}{(\Delta x)^2}
\end{equation}

Starting from an initial box density profile, the 1D diffusion equation (3) is numerically solved based on equation (4). The density profile at later times is plotted in figure~\ref{diffusion}. The density are normally distributed, with different standard deviation $\sigma$. The value of $\sigma$ at time t can be extracted from the maximum value of density $u_{max}$ at $x=0$.

\begin{figure}[h!]
\centering
\includegraphics[width=0.5\textwidth]{diffusion.pdf}
\caption{Solution of 1D diffusion equation at time 0, 0.2, 0.4, 0.6, 0.8, 1.0.}
\label{diffusion}
\end{figure}

\begin{align*}
u_{max} (t) & = \rho(0,t) = \frac{1}{\sqrt{2\pi \sigma^2 (t)}} \\
\Rightarrow \sigma(t) & = \frac{1}{\sqrt{2\pi u_{max}^2 (t)}}
\end{align*}

The maximum values $u_{max}$ and derived $sigma$ at time 0.2, 0.4, 0.6, 0.8, 1.0 are listed in table~\ref{umax}. The $\sigma (t)$ versus time t is plotted (log scale) in figure~\ref{sigma} with a linear fit line, which has a scope of 0.478. This indicates that $\sigma(t) \propto \sqrt{t}$.

\begin{table}[!ht]
\begin{center}
\caption{$u_{max}$ at time 0.2, 0.4, 0.6, 0.8, 1.0.}
\begin{tabular}{ l | c | c | c | c | c }
\hline
t & 0.2 & 0.4 & 0.6 & 0.8 & 1.0 \\ \hline
$u_{max}$ & 22.1449 & 16.0971 & 13.2666 & 11.5433 & 10.3539 \\ \hline
$\sigma$ [$\times 10^{-2}$] & 1.8015 & 2.4783 & 3.0071 & 3.4560 & 3.8531 \\ \hline
\end{tabular}
\end{center}
\label{umax}
\end{table}

\begin{figure}[h!]
\centering
\includegraphics[width=0.5\textwidth]{sigma.pdf}
\caption{$\sigma$ versus t (log plot).}
\label{sigma}
\end{figure}

\section{Mixing of two Gases}

\begin{figure}[h!]
\centering
\begin{subfigure}[b]{0.32\textwidth}
%\includegraphics[width=\textwidth]{snapshot_0.pdf}
%\caption{$n = 0$}
\end{subfigure}
\begin{subfigure}[b]{0.32\textwidth}
%\includegraphics[width=\textwidth]{snapshot_1.pdf}
%\caption{$n = $}
\end{subfigure}
\begin{subfigure}[b]{0.32\textwidth}
%\includegraphics[width=\textwidth]{snapshot_2.pdf}
%\caption{$n = $}
\end{subfigure}
\begin{subfigure}[b]{0.32\textwidth}
%\includegraphics[width=\textwidth]{snapshot_3.pdf}
%\caption{$n = $}
\end{subfigure}
\begin{subfigure}[b]{0.32\textwidth}
%\includegraphics[width=\textwidth]{snapshot_4.pdf}
%\caption{$n = $}
\end{subfigure}
\begin{subfigure}[b]{0.32\textwidth}
%\includegraphics[width=\textwidth]{snapshot_5.pdf}
%\caption{$n = $}
\end{subfigure}
\label{densities}
\caption{Gases mixing configurations after different number of steps.}
\end{figure}

\begin{figure}[h!]
\centering
\begin{subfigure}[b]{0.32\textwidth}
\includegraphics[width=\textwidth]{densities_0.pdf}
%\caption{$n = 0$}
\end{subfigure}
\begin{subfigure}[b]{0.32\textwidth}
\includegraphics[width=\textwidth]{densities_1.pdf}
%\caption{$n = $}
\end{subfigure}
\begin{subfigure}[b]{0.32\textwidth}
\includegraphics[width=\textwidth]{densities_2.pdf}
%\caption{$n = $}
\end{subfigure}
\begin{subfigure}[b]{0.32\textwidth}
\includegraphics[width=\textwidth]{densities_3.pdf}
%\caption{$n = $}
\end{subfigure}
\begin{subfigure}[b]{0.32\textwidth}
\includegraphics[width=\textwidth]{densities_4.pdf}
%\caption{$n = $}
\end{subfigure}
\begin{subfigure}[b]{0.32\textwidth}
\includegraphics[width=\textwidth]{densities_5.pdf}
%\caption{$n = $}
\end{subfigure}
\label{densities}
\caption{Linear population densities after different number of steps.}
\end{figure}

\begin{figure}[h!]
\centering
\includegraphics[width=0.5\textwidth]{densities_average.pdf}
\caption{Linear population densities averaged over 100 trials.}
\label{average100}
\end{figure}

\section{Conclusions}

\begin{thebibliography}{9}
\bibitem{notes}
Notes of \textbf{\textit{Computational Physics}} by Prof. S.A. Bass.
\end{thebibliography}

\end{document}
