\documentclass{article}
\usepackage{geometry}
\usepackage{graphicx}
\usepackage{caption}
\usepackage{subcaption}
\usepackage{amsmath}
\geometry{left=1.8cm,right=1.8cm,top=1.5cm,bottom=1.5cm}
\parindent=0pt

\begin{document}
\title{\textbf{Computational Physics Group Assignment 1}}
\author{Christopher Flower, Long Li, Shen Yan, Wenzhe Yu}
\maketitle

\section{2D Random Walk}

In this problem, we study the properties of a 2-dimensional random walk. We create random walks of different steps: from 3 to 100 steps. 
In order to study the average of X-coordinate $<x_n>$, of the square of X-coordinate $<(X_n)^2>$, and of the square distance $<(r_n)^2>$ of random walks with different steps $n$, we, for one random walk with a certain step, average over $10^4$ different walks with same steps. \\
\\
As we can see from the figure, $<x_n>$ is oscillating around the zero. Indeed, $<x_n>$ should be zero theoretically. $<(X_n)^2>$, however, is increasing with steps, and the linear coefficient is approximately $\frac{1}{2}$. \\
\\
Since this is a symmetric random walk, y-direction should possess the same property with x-direction, so we should anticipate $<(r_n)^2>$ to be also linearly increasing with steps, with the coefficient being $1$. And it is indeed the case shown on the figure.

\begin{figure}[h!]
\centering
\includegraphics[width = 0.8\textwidth]{xaverage.pdf}
\caption{  $<x_n>$ vs steps n}
\label{xaverage}
\end{figure}

\begin{figure}[h!]
\centering
\includegraphics[width = 0.8\textwidth]{x2average.pdf}
\caption{ $<(X_n)^2>$ vs steps n}
\label{x2average}
\end{figure}

\begin{figure}[h!]
\centering
\includegraphics[width = 0.8\textwidth]{r2average.pdf}
\caption{ $<(r_n)^2>$ vs steps n}
\label{r2average}
\end{figure}



\section{Diffusion Equation}
\textbf{a)} Consider the 1D Normal Distribution

\begin{equation}
\rho(x,t) = \frac{1}{\sqrt{2\pi \sigma^2 (t)}} e^{-\frac{x^2}{2\sigma^2(t)}}
\end{equation}

The spatial expectation value $<x^2(t)>$ can be computed by

\begin{align*}
<x^2(t)> & = \int_{-\infty}^{\infty} x^2 \rho(x,t) dt \\
& = \frac{1}{\sqrt{2\pi \sigma^2 (t)}} \int_{-\infty}^{\infty} x^2 e^{-\frac{x^2}{2\sigma^2(t)}} dx \\
& = \frac{1}{\sqrt{2\pi \sigma^2 (t)}} \frac{1}{2} (\sqrt{2} \sigma(t))^3 \sqrt{\pi} \\
& = \sigma^2 (t)
\end{align*}

Here $\int_{-\infty}^{\infty} x^{2k} e^{-\frac{x^2}{a^2}} dx = \frac{(2k+1)!!}{(2k+1)2^k} a^{2k+1} \sqrt{\pi}\ (k=0,1,2,...)$ is used.\\
\\
\textbf{b)} The diffusion equation with a constant diffusion coefficient has the following form:

\begin{equation}
\frac{\partial u (r,t)}{\partial t} = D \nabla^2 u(r,t)
\end{equation}

In one-dimensional case, equation (2) becomes

\begin{equation}
\frac{\partial u (r,t)}{\partial t} = D \frac{\partial^2 u(x,t)}{\partial x^2}
\end{equation}

The first derivative in time and second derivative in space can be approximated by the finite difference:

\begin{align*}
\frac{\partial u (r,t)}{\partial t} & = \frac{u(x,t+\Delta t) - u(x,t)}{\Delta t} \\
\frac{\partial^2 u(x,t)}{\partial x^2} & = \frac{u(x+\Delta x,t)+u(x-\Delta x,t)-2u(x,t)}{(\Delta x)^2}
\end{align*}

Then equation (3) can be rewritten as

\begin{equation*}
\frac{u(x,t+\Delta t) - u(x,t)}{\Delta t} = D \cdot \frac{u(x+\Delta x,t)+u(x-\Delta x,t)-2u(x,t)}{(\Delta x)^2}
\end{equation*}

or equivalently,

\begin{equation}
u(x,t+\Delta t) = u(x,t) + D \cdot \Delta t \cdot \frac{u(x+\Delta x,t)+u(x-\Delta x,t)-2u(x,t)}{(\Delta x)^2}
\end{equation}

Starting from an initial box density profile, the 1D diffusion equation (3) is numerically solved based on equation (4). The density profile at later times is plotted in figure~\ref{diffusion}. The density are normally distributed, with different standard deviation $\sigma$. The value of $\sigma$ at time t can be extracted from the maximum value of density $u_{max}$ at $x=0$.

\begin{figure}
\centering
\includegraphics[width=0.5\textwidth]{2b_1.pdf}
\caption{Solution of 1D diffusion equation at time 0, 0.2, 0.4, 0.6, 0.8, 1.0.}
\label{diffusion}
\end{figure}

\begin{align*}
u_{max} (t) & = \rho(0,t) = \frac{1}{\sqrt{2\pi \sigma^2 (t)}} \\
\Rightarrow \sigma(t) & = \frac{1}{\sqrt{2\pi u_{max}^2 (t)}}
\end{align*}

The maximum values $u_{max}$ and derived $sigma$ at time 0.2, 0.4, 0.6, 0.8, 1.0 are listed in table~\ref{umax}. The $\sigma (t)$ versus time t is plotted (log scale) in figure~\ref{sigma} with a linear fit line, which has a scope of 0.478. This indicates that $\sigma(t) \propto \sqrt{t}$.

\begin{table}[!ht]
\begin{center}
\caption{$u_{max}$ at time 0.2, 0.4, 0.6, 0.8, 1.0.}
\begin{tabular}{ l | c | c | c | c | c }
\hline
t & 0.2 & 0.4 & 0.6 & 0.8 & 1.0 \\ \hline
$u_{max}$ & 22.1449 & 16.0971 & 13.2666 & 11.5433 & 10.3539 \\ \hline
$\sigma$ [$\times 10^{-2}$] & 1.8015 & 2.4783 & 3.0071 & 3.4560 & 3.8531 \\ \hline
\end{tabular}
\end{center}
\label{umax}
\end{table}

\begin{figure}
\centering
%\begin{subfigure}[b]{0.45\textwidth}
%\includegraphics[width=\textwidth]{b11.png}
%\caption{10 subdivisions}
%\end{subfigure}
\includegraphics[width=0.5\textwidth]{2b_2.pdf}
\caption{$\sigma$ versus t (log plot).}
\label{sigma}
\end{figure}

\section{Mixing of two Gases}

\end{document}
